\documentclass[12pt, psamsfonts]{amsart}

%-------Packages---------
\usepackage{amssymb,amsfonts}
\usepackage{semantic}
\usepackage{fullpage}
\usepackage{tikz-cd}
\usepackage{todonotes}
\usepackage{physics}
\usepackage[all,arc]{xy}
\usepackage{enumerate}
\usepackage{enumitem}
\usepackage{mathrsfs}
\usepackage{theoremref}
\usepackage{graphicx}
\usepackage[bookmarks]{hyperref}

\usepackage{amsthm}
\makeatletter
\def\th@plain{%
  \thm@notefont{}% same as heading font
  \itshape % body font
}
\def\th@definition{%
  \thm@notefont{}% same as heading font
  \normalfont % body font
}
\makeatother

%--------Theorem Environments--------
%theoremstyle{plain} --- default
\newtheorem{thm}{Theorem}[section]
\newtheorem{cor}[thm]{Corollary}
\newtheorem{prop}[thm]{Proposition}
\newtheorem{lem}[thm]{Lemma}
\newtheorem{conj}[thm]{Conjecture}
\newtheorem{quest}[thm]{Question}

\theoremstyle{definition}
\newtheorem{defn}[thm]{Definition}
\newtheorem{defns}[thm]{Definitions}
\newtheorem{con}[thm]{Construction}
\newtheorem{exmp}[thm]{Example}
\newtheorem{exmps}[thm]{Examples}
\newtheorem{notn}[thm]{Notation}
\newtheorem{notns}[thm]{Notations}
\newtheorem{addm}[thm]{Addendum}
\newtheorem*{exer}{Exercise}

\theoremstyle{remark}
\newtheorem{rem}[thm]{Remark}
\newtheorem{rems}[thm]{Remarks}
\newtheorem{warn}[thm]{Warning}
\newtheorem{sch}[thm]{Scholium}

\makeatletter
\makeatother
\numberwithin{equation}{section}

\bibliographystyle{plain}

\begin{document}

\title{Stellar Consensus Protocol(Sub-notes)}
\author{Hidenori Shinohara}

\begin{abstract}
    This is a collection of things that are related to SCP, but not enough to make it to the main notes.
\end{abstract}

\maketitle

\tableofcontents

\section{Prerequisites}

\begin{thm}
    Let $f \in \mathbb{N}$ be given.
    Consider a system with $2f + 1$ nodes such that any $f + 1$ of them constitute a quorum.
    Then $f$ is the maximum number of fail-stop failures that the system can survive while
    \begin{itemize}
        \item
            maintaining safety (i.e., no two well-behaved nodes will agree on contradictory statements)
        \item
            maintaining liveness (i.e., all well-behaved nodes can agree on any valid statements)
    \end{itemize}
\end{thm}

\begin{proof}
    Let $b$ be the number of fail-stop failures and assume $b \leq f$.
    Then there are $2f + 1 - b \geq f + 1$ well-behaved nodes.
    Thus each well-behaved node can find a quorum it belongs to that only consists of well-behaved nodes because there are at least $f + 1$ well-behaved nodes.
    This shows the liveness of the system.

    Let $v_1, v_2$ be two well-behaved nodes and suppose they agreed on contradictory statements $a_1, a_2$, respectively.
    Let $Q_1, Q_2$ be the quorums that convinced $v_1, v_2$ on $a_1, a_2$, respectively.
    By the pigeonhole principle, $\abs{Q_1 \cap Q_2} \geq 1$.
    In other words, there exists a node that agreed on both $a_1$ and $a_2$.
    This is a contradiction because we assumed that nodes are either well-behaved or fail-stop.
    Therefore, two well-behaved nodes cannot agree on contradictory statements.
\end{proof}

\begin{thm}
    Let $f \in \mathbb{N}$ be given.
    Consider a system with $3f + 1$ nodes such that any $2f + 1$ of them constitute a quorum.
    Then $f$ is the maximum number of Byzantine failures that the system can survive while
    \begin{itemize}
        \item
            maintaining safety (i.e., no two well-behaved nodes will agree on contradictory statements)
        \item
            maintaining liveness (i.e., all well-behaved nodes can agree on any valid statements)
    \end{itemize}
\end{thm}

\begin{proof}
    Let $b$ be the number of Byzantine failures and assume $b \leq f$.

    Then there are $3f + 1 - b \geq 2f + 1$ well-behaved nodes.
    Thus each well-behaved node can find a quorum it belongs to that only consists of well-behaved nodes because there are at least $2f + 1$ well-behaved nodes.
    This shows the liveness of the system.

    Let $v_1, v_2$ be two well-behaved nodes and suppose they agreed on contradictory statements $a_1, a_2$, respectively.
    Let $Q_1, Q_2$ be the quorums that convinced $v_1, v_2$ on $a_1, a_2$, respectively.
    By the pigeonhole principle, $\abs{Q_1 \cap Q_2} \geq f + 1$.
    In other words, at least $f + 1$ nodes agreed on both $a_1$ and $a_2$.
    This is a contradiction because we assumed that there are at most $b \leq f$ Byzantine failures.
    Therefore, two well-behaved nodes cannot agree on contradictory statements.

    We have shown that the system can survive while maintaining safety and liveness with $b$ Byzantine failures.
    We will now show that $f$ is the largest number with such a property.

    Let $b > f$ and assume that the system experiences $b$ Byzantine failures.
    By the pigeonhole principle, each quorum contains a Byzantine failure.
    We cannot guarantee liveness because Byzantine nodes can all disagree with everything. 
\end{proof}

\end{document}

\documentclass[12pt, psamsfonts]{amsart}

%-------Packages---------
\usepackage{amssymb,amsfonts}
\usepackage{semantic}
\usepackage{fullpage}
\usepackage{tikz-cd}
\usepackage{todonotes}
\usepackage{physics}
\usepackage[all,arc]{xy}
\usepackage{enumerate}
\usepackage{enumitem}
\usepackage{mathrsfs}
\usepackage{theoremref}
\usepackage{graphicx}
\usepackage[bookmarks]{hyperref}

%--------Theorem Environments--------
%theoremstyle{plain} --- default
\newtheorem{thm}{Theorem}[section]
\newtheorem{cor}[thm]{Corollary}
\newtheorem{prop}[thm]{Proposition}
\newtheorem{lem}[thm]{Lemma}
\newtheorem{conj}[thm]{Conjecture}
\newtheorem{quest}[thm]{Question}

\theoremstyle{definition}
\newtheorem{defn}[thm]{Definition}
\newtheorem{defns}[thm]{Definitions}
\newtheorem{con}[thm]{Construction}
\newtheorem{exmp}[thm]{Example}
\newtheorem{exmps}[thm]{Examples}
\newtheorem{notn}[thm]{Notation}
\newtheorem{notns}[thm]{Notations}
\newtheorem{addm}[thm]{Addendum}
\newtheorem*{exer}{Exercise}

\theoremstyle{remark}
\newtheorem{rem}[thm]{Remark}
\newtheorem{rems}[thm]{Remarks}
\newtheorem{warn}[thm]{Warning}
\newtheorem{sch}[thm]{Scholium}

\makeatletter
\makeatother
\numberwithin{equation}{section}

\bibliographystyle{plain}

\begin{document}

\title{Stellar Consensus Protocol}
\author{Hidenori Shinohara}

\begin{abstract}
    This is my personal notes on the Stellar consensus protocol.
    This roughly follows the structure of the white paper on https://www.stellar.org/.
\end{abstract}

\maketitle

\tableofcontents

\section{Prerequisites}

\begin{thm}
    Let $f \in \mathbb{N}$ be given.
    Consider a system with $2f + 1$ nodes such that any $f + 1$ of them constitute a quorum.
    Then $f$ is the maximum number of fail-stop failures that the system can survive while
    \begin{itemize}
        \item
            maintaining safety (i.e., no two well-behaved nodes will agree on contradictory statements)
        \item
            maintaining liveness (i.e., all well-behaved nodes can agree on any valid statements)
    \end{itemize}
\end{thm}

\begin{proof}
    Let $b$ be the number of fail-stop failures and assume $b \leq f$.
    Then there are $2f + 1 - b \geq f + 1$ well-behaved nodes.
    Thus each well-behaved node can find a quorum it belongs to that only consists of well-behaved nodes because there are at least $f + 1$ well-behaved nodes.
    This shows the liveness of the system.

    Let $v_1, v_2$ be two well-behaved nodes and suppose they agreed on contradictory statements $a_1, a_2$, repspectively.
    Let $Q_1, Q_2$ be the quorums that convinced $v_1, v_2$ on $a_1, a_2$, respectively.
    By the pigeonhole principle, $\abs{Q_1 \cap Q_2} \geq 1$.
    In other words, there exists a node that agreed on both $a_1$ and $a_2$.
    This is a contradiction because we assumed that nodes are either well-behaved or fail-stop.
    Therefore, two well-behaved nodes cannot agree on contradictory statements.
\end{proof}

\begin{thm}
    Let $f \in \mathbb{N}$ be given.
    Consider a system with $3f + 1$ nodes such that any $2f + 1$ of them constitute a quorum.
    Then $f$ is the maximum number of Byzantine failures that the system can survive while
    \begin{itemize}
        \item
            maintaining safety (i.e., no two well-behaved nodes will agree on contradictory statements)
        \item
            maintaining liveness (i.e., all well-behaved nodes can agree on any valid statements)
    \end{itemize}
\end{thm}

\begin{proof}
    Let $b$ be the number of Byzantine failures and assume $b \leq f$.

    Then there are $3f + 1 - b \geq 2f + 1$ well-behaved nodes.
    Thus each well-behaved node can find a quorum it belongs to that only consists of well-behaved nodes because there are at least $2f + 1$ well-behaved nodes.
    This shows the liveness of the system.

    Let $v_1, v_2$ be two well-behaved nodes and suppose they agreed on contradictory statements $a_1, a_2$, repspectively.
    Let $Q_1, Q_2$ be the quorums that convinced $v_1, v_2$ on $a_1, a_2$, respectively.
    By the pigeonhole principle, $\abs{Q_1 \cap Q_2} \geq f + 1$.
    In other words, at least $f + 1$ nodes agreed on both $a_1$ and $a_2$.
    This is a contradiction because we assumed that there are at most $b \leq f$ Byzantine failures.
    Therefore, two well-behaved nodes cannot agree on contradictory statements.

    We have shown that the system can survive while maintaining safety and liveness with $b$ Byzantine failures.
    We will now show that $f$ is the largest number with such a property.

    Let $b > f$ and assume that the system experiences $b$ Byzantine failures.
    By the pigeonhole principle, each quorum contains a Byzantine failure.
    We cannot guarantee liveness because Byzantine nodes can all disagree with everything. 
\end{proof}

\section{Basic Properties of Quorums}

\begin{defn}
    Let $V$ be a set and $Q:V \rightarrow 2^{2^V} \setminus \{ \emptyset \}$ be a function such that $\forall v \in V, \forall q \in Q(v), v \in q$.
    Then we call the pair $\ev{V, Q}$ a federated Byzantine agreement system, or FBAS for short.
\end{defn}

\begin{rem}
    For each node $v$, $Q(v)$ is a set of subsets of $V$.
    For instance, node $v_1$ may trust $v_2, v_3, v_4$ and may have $Q(v_1) = \{ \{ v_1, v_2, v_3, v_4 \} \} \subset 2^V$.

    We explicitly exclude $\{ \emptyset \}$ from the co-domain because we want $Q(v) \ne \emptyset$ for all $v \in V$.
    This is necessary because if $Q(v) = \emptyset$ for some $v \in V$, it satisfies $\forall q \in Q(v), v \in q$.
\end{rem}

\begin{defn}
    Let $\ev{V, Q}$ be an FBAS\@.
    $U \subset V$ is called a quorum if and only if $\forall v \in U, \exists q \in Q(v), q \subset U$.
\end{defn}

\begin{thm}\label{union_quorums}
    In an FBAS $\ev{V, Q}$, the union of two quorums is a quorum.
\end{thm}

\begin{proof}
    Let $U_1, U_2$ be two quorums.
    Let $v \in U_1 \cup U_2$.
    Then $v \in U_i$ for $i = 1$ or $i = 2$.
    Then $q \subset U_i$ for some $q \in Q(v)$.
    Therefore, $q \subset U_1 \cup U_2$, so $U_1 \cup U_2$ is indeed a quorum.
\end{proof}

\begin{cor}
    The set of quorums of a given FBAS is closed under union.
\end{cor}

\begin{thm}
    In an FBAS $\ev{V, Q}$, $V$ is a quorum.
\end{thm}

\begin{proof}
    For any $v \in V$, for any $q \in Q(v)$, $q \subset V$.
    Therefore, $V$ is indeed a quorum.
\end{proof}

\begin{exmp}
    One might wonder if the intersection of quorums is always a quorum.
    However, this is not true in general.

    Let $V = \{ v_1, \ldots, v_4 \}$ and
    \begin{itemize}
        \item
            $Q(v_1) = \{ \{ v_1, v_2, v_3 \}, \{ v_1, v_2, v_4 \}, \{ v_1, v_3, v_4 \} \}$,
        \item
            $\vdots$
        \item
            $Q(v_4) = \{ \{ v_1, v_2, v_4 \}, \{ v_1, v_3, v_4 \}, \{ v_2, v_3, v_4 \} \}$.
    \end{itemize}

    In other words, $Q(v_i) = \{ U \subset V \mid \abs{U} = 3, v_i \in U \}$.

    Then $U_1 = \{ v_1, v_2, v_3 \}$ is a quorum, and $U_2 = \{ v_2, v_3, v_4 \}$ is a quorum.
    However, $U_1 \cap U_2 = \{ v_2, v_3 \}$ is not a quorum because the size of any quorum slice is 3.
\end{exmp}

\begin{defn}
    Let $\ev{V, Q}$ be an FBAS\@.
    We say $\ev{V, Q}$ enjoys quorum intersection if and only if for any pair of quorums $U_1, U_2$, $U_1 \cap U_2 \ne \emptyset$.
\end{defn}

\begin{defn}\label{delete_fbas}
    Let $\ev{V, Q}$ be an FBAS and $B \subset V$.
    Then the FBAS $\ev{V, Q}^B$ is defined to be $\ev{V \setminus B, Q^B}$ where $\forall v \in V \setminus B, Q^B(v) = \{ q \setminus B \mid q \in Q(v) \}$.
\end{defn}

\begin{rem}
    One may think that this is related to fail-stop behaviors where $B$ is the set of nodes that stopped responding.
    In general, however, this is not true as we also remove nodes from quorum slices.
    One can think of this as the alternate universe where nodes from $B$ simply did not even exist from the beginning.
\end{rem}

\begin{thm}
    Definition~\ref{delete_fbas} is well-defined.
    In other words, if $\ev{V, Q}$ is an FBAS and $B \subset V$, then $\ev{V, Q}^B$ is an FBAS\@.
\end{thm}

\begin{proof}
    Let $v \in V \setminus B, q' \in Q^B(v)$ be given.
    Then $q' = q \setminus B$ for some $q \in Q(v)$.
    By the definition of an FBAS, $v \in q$.
    Since $v \notin B$, $v \in q \setminus B = q'$.
    Therefore, $\ev{V, Q}^B$ is an FBAS\@.
\end{proof}

\begin{thm}\label{quorum_intersection_projected_system}
    Let $U$ be a quorum in FBAS $\ev{V, Q}$, let $B \subset V$ be a set of nodes, and let $U' = U \setminus B$.
    If $U' \ne \emptyset$, then $U'$ is a quorum in $\ev{V, Q}^B$.
\end{thm}

\begin{proof}
    Since $U' \ne \emptyset$, it suffices to show that $\forall v \in U', \exists q \in Q^B(v), q \subset U'$.
    Let $v \in U'$.
    Then $v \in U$.
    Since $U$ is a quorum in $\ev{V, Q}$, we can find $q \in Q(v)$ such that $q \subset U$.
    Then $q' = q \setminus B \in Q^B(v)$, and $q' = q \setminus B \subset U \setminus B = U'$.
    Therefore, $U'$ is a quorum in $\ev{V, Q}^B$.
\end{proof}

\begin{rem}
    One can think of this theorem as ``\textit{A quorum in the `original' universe is a quorum in the `alternate' universe.}"
\end{rem}

\begin{defn}
    Let $\ev{V, Q}$ be an FBAS and $B \subset V$ be a set of nodes.
    We say $\ev{V, Q}$ enjoys quorum intersection despite $B$ if and only if $\ev{V, Q}^B$ enjoys quorum intersection.
\end{defn}

\begin{rem}
    Quorum intersection despite $B$ is related to system-level safety when nodes in $B$ act arbitrarily.
    For instance, suppose $\ev{V, Q}$ is an FBAS, $B$ is the set of all ill-behaved nodes, and $\ev{V, Q}$ enjoys quorum intersection despite $B$.
    Suppose two well-behaved nodes $v_1, v_2$ agree with contraditory statements $a_1, a_2$ in quorums $q_1, q_2$, respectively.
    Then $q_1 \cap q_2 \ne \emptyset$ have well-behaved nodes who agreed with both $a_1$ and $a_2$.
    This is a contradiction because a well-behaved node cannot contradict itself.
    This example illustrates how the concept of quorum intersection despite $B$ is related to system-level safety when nodes in $B$ experience Byzantine failures.
\end{rem}

\begin{defn}\label{def_quorum_availability}
    Let $\ev{V, Q}$ be an FBAS and $B \subset V$ be a set of nodes.
    We say $\ev{V, Q}$ enjoys quorum availability despite $B$ if and only if $V \setminus B$ is a quorum in $\ev{V, Q}$ or $B = V$.
\end{defn}

\begin{thm}\label{quorum_availability_equivalence_condition}
    Let $\ev{V, Q}$ be an FBAS and $B \subset V$.
    $\ev{V, Q}$ enjoys quorum availability despite $B$ if and only if $\forall v \in V \setminus B$, there exists a quorum $U_v$ such that $v \in U_v \subset (V \setminus B)$.
\end{thm}

\begin{proof}
    If $V = B$, we are done.
    Suppose otherwise.
    \begin{align*}
        \forall v \in V \setminus B, \exists \text{a quorum $U_v$}, v \in U_v \subset (V \setminus B)
            &\implies \bigcup_{v \in V \setminus B} U_v \text{ is a quorum in $\ev{V, Q}$} \\
            &\implies \text{$V \setminus B$ is a quorum in $\ev{V, Q}$}
    \end{align*}
    by Theorem~\ref{union_quorums}.
    On the other hand, if $V \setminus B$ is a quorum in $\ev{V, Q}$, then $\forall v \in V \setminus B, \exists \text{a quorum $U_v$}, v \in U_v \subset (V \setminus B)$ because we can let $U_v = V \setminus B$ for each $v$.
\end{proof}

\begin{rem}
    Theorem~\ref{quorum_availability_equivalence_condition} shows that $\ev{V, Q}$ enjoys quorum availability despite $B$ if all nodes in $V \ B$ can find a quorum without $B$.
    This is related to liveness of the system.
    If $\ev{V, Q}$ enjoys quorum availability despite $B$, then regardless of what happens to nodes in $B$, nodes in $V \setminus B$ can keep going.
\end{rem}

\begin{defn}\label{def_v_blocking}
    Let $\ev{V, Q}$ be an FBAS\@.
    Let $v \in V$.
    A subset $B \subset V$ is called $v$-blocking if and only if $\forall q \in Q(v), q \cap B \ne \emptyset$.
\end{defn}

\begin{rem}
    Intuitively, if a subset $B \subset V$ is $v$-blocking, then one may think of it as ``\textit{$v$ can't really get by without $B$}."
    The following theorem can be interpreted as ``\textit{If $v$ can't get by without $B$, $v$ can't get by without $C$ for any $C \supset B$}."
\end{rem}

\begin{thm}\label{basic_prop_v_blocking}
    Let $\ev{V, Q}$ be an FBAS\@.
    Let $v \in V$.
    Then
    \begin{itemize}
        \item
            The union of two $v$-blocking sets is $v$-blocking.
        \item
            Any superset of a $v$-blocking set is $v$-blocking.
    \end{itemize}
\end{thm}

\begin{proof}
    It suffices to only prove the second statement.
    If $B \subset B'$ and $B$ is $v$-blocking, $q \cap B' \supset q \cap B \ne \emptyset$ for any $q \in Q(v)$.
\end{proof}

\begin{thm}\label{v_blocking_delete}
    Let $\ev{V, Q}$ be an FBAS\@.
    Let $A \subsetneq V$ and $U_1, U_2$ be a partition of $V \setminus A$.
    Let $v \in U_1$.
    If $U_2$ is not $v$-blocking in $\ev{V, Q}$, then $U_2$ is not $v$-blocking in $\ev{V, Q}^A$.
\end{thm}

\begin{proof}
    Since $U_2$ is not $v$-blocking in $\ev{V, Q}$, there exists $q_v \in Q(v)$ such that $q_v \cap U_2 = \emptyset$.
    \begin{align*}
        (q_v \setminus A) \cap U_2
            &= (q_v \cap U_2) \setminus (A \cap U_2) \\
            &= q_v \cap U_2 = \emptyset.
    \end{align*}
    Thus $U_2$ is not $v$-blocking in $\ev{V, Q}^A$.
\end{proof}

\begin{thm}\label{quorum_availability_v_blocking}
    Let $\ev{V, Q}$ be an FBAS\@.
    Let $B \subset V$.
    $\ev{V, Q}$ enjoys quorum availability despite $B$ if and only if $B$ is not $v$-blocking for any $v \in V \setminus B$.
\end{thm}

\begin{proof}
    \begin{align*}
        \forall v \in V \setminus B, \neg(\text{$B$ is $v$-blocking})
            &\iff \forall v \in V \setminus B, \neg(\forall q \in Q(v), q \cap B \ne \emptyset) \\
            &\iff \forall v \in V \setminus B, \exists q \in Q(v), q \cap B = \emptyset \\
            &\iff \forall v \in V \setminus B, \exists q \in Q(v), q \subset V \setminus B \\
            &\iff \text{$V = B$ or $V \setminus B$ is a quorum in $\ev{V, Q}$} \\
            &\iff \text{$\ev{V, Q}$ enjoys quorum availability despite $B$}
    \end{align*}
\end{proof}

\section{Dispensable Sets}

\begin{defn}\label{def_dset}
    Let $\ev{V, Q}$ be an FBAS and $B \subset V$ be a set of nodes.
    $B$ is called a dispensable set, or DSet, if and only if $\ev{V, Q}$ enjoys quorum intersection and availability despite $B$.
\end{defn}

\begin{defn}\label{def_intact_befouled}
    Let $\ev{V, Q}$ be an FBAS and $v \in V$.
    $v$ is said to be intact if and only if there exists a DSet $B$ containing all ill-behaved nodes and $v \notin B$.
    $v$ is said to be befouled if and only if $v$ is not intact.
\end{defn}

\begin{thm}\label{intersection_dset}
    If $B_1$ and $B_2$ are DSets in an FBAS $\ev{V, Q}$ enjoying quorum intersection, then $B = B_1 \cap B_2$ is a DSet, too.
\end{thm}

\begin{proof}
	If $B_1 = V$ or $B_2 = V$, then we are done.
	Suppose otherwise.

	For any $v \in V$,

	\begin{align*}
	  v \in V \setminus B
		&\iff v \in V \land v \notin B \\
		&\iff v \in V \land (v \notin B_1 \lor v \notin B_2) \\
		&\iff (v \in V \land v \notin B_1) \lor (v \in V \land v \notin B_2) \\
		&\iff (v \in (V \setminus B_1)) \lor (v \in (V \setminus B_2)) \\
		&\iff v \in ((V \setminus B_1) \cup (V \setminus B_2)).
	\end{align*}

	Thus, $V \setminus B = (V \setminus B_1) \cup (V \setminus B_2)$.
	By the definition of a DSet, $V \setminus B_1$ and $V \setminus B_2$ are both quorums in $\ev{V, Q}$.
	By Theorem~\ref{union_quorums}, $V \setminus B$ is a quorum in $\ev{V, Q}$.

	We must now show quorum intersection despite $B$.
	Let $U_a, U_b$ be quorums in $\ev{V, Q}^B$.

	\begin{itemize}
		\item
            $U_a \setminus B_1$ is a quorum in ${(\ev{V, Q}^B)}^{B_1} = \ev{V, Q}^{B_1}$ by Theorem~\ref{delete_fbas}.
		\item
			Similarly, $U_b \setminus B_1$ is a quorum in $\ev{V, Q}^{B_1}$, and $U_a \setminus B_2$ and $U_b \setminus B_2$ are both quorums in $\ev{V, Q}^{B_2}$.
	\end{itemize}

	\begin{align*}
	  (U_a \setminus B_1) \cup (U_a \setminus B_2)
		&= U_a \setminus (B_1 \cap B_2) \\
		&= U_a \setminus B \\
		&= U_a
	\end{align*}

	because $U_a$ is a quorum in $\ev{V, Q}^B$.
	In other words, $(U_a \setminus B_1) \cup (U_a \setminus B_2) \ne \emptyset$.
	Similarly, $(U_b \setminus B_1) \cup (U_b \setminus B_2) \ne \emptyset$.

	Without loss of generality, assume that $U_a \setminus B_1 \ne \emptyset$.
	\begin{itemize}
		\item
			$V \setminus B_1$ is a quorum in $\ev{V, Q}$ because $B_1$ is a DSet.
			Similarly, $V \setminus B_2$ is a quorum in $\ev{V, Q}$.
			Because $\ev{V, Q}$ enjoys quorum intersection, $(V \setminus B_1) \cap (V \setminus B_2) \ne \emptyset$.
			In other words, $(V \setminus B_2) \setminus B_1$ is a quorum.
			By Theorem~\ref{delete_fbas}, $(V \setminus B_2) \setminus B_1$ is a quorum in $\ev{V, Q}^{B_1}$.
		\item
            $U_a \setminus B_1$ is a quorum in ${(\ev{V, Q}^B)}^{B_1} = {\ev{V, Q}}^{B_1}$ for the same reason.
	\end{itemize}
	Because $B_1$ is a DSet in $\ev{V, Q}$, $\ev{V, Q}^{B_1}$ enjoys quorum intersection.
	Therefore, $(U_a \setminus B_1) \cap ((V \setminus B_2) \setminus B_1) \ne \emptyset$.
	\begin{align*}
	  (U_a \setminus B_1) \cap ((V \setminus B_2) \setminus B_1)
		&= (U_a \cap (V \setminus B_2)) \setminus B_1 \\
		&\subset U_a \cap (V \setminus B_2) \\
		&= (U_a \cap V) \setminus B_2 \\
		&= U_a \setminus B_2.
	\end{align*}
	Thus, $U_a \setminus B_2 \ne \emptyset$.
	Using the same argument, we can show that $U_b \setminus B_1 \ne \emptyset$ and $U_b \setminus B_2 \ne \emptyset$.
	Since $U_a \setminus B_1$ and $U_b \setminus B_1$ are quorums in $\ev{V, Q}^{B_1}$ and $B_1$ is a DSet, $(U_a \setminus B_1) \cap (U_b \setminus B_1) \ne \emptyset$ by the definition of a DSet.
	This implies $(U_a \cap U_b) \setminus B_1 \ne \emptyset$.
	Therefore, $U_a \cap U_b \ne \emptyset$.
\end{proof}

\begin{thm}\label{befouled_dset}
	In an FBAS with quorum intersection, the set of befouled nodes is a DSet.
\end{thm}

\begin{proof}
	Let $\ev{V, Q}$ be an FBAS with quorum intersection.
	Let $B$ be the intersection of all DSets that contain all ill-behaved nodes.
	By Theorem~\ref{intersection_dset}, $B$ is a DSet.

	\begin{itemize}
		\item
			Case 1: $v \in B$.
			Then there exists no DSet $B_v$ such that $B_v$ contains all ill-behaved nodes and $v \notin B_v$.
			Therefore, $v$ is not an intact node.
			In other words, $v$ is a befouled node.
		\item
			Case 2: $v \notin B$.
			Then there exists a DSet $B_v$ that contains all ill-behaved nodes and $v \notin B_v$.
			In other words, $v$ is intact and thus $v$ is not a befouled node.
	\end{itemize}

	Therefore, $B$ is precisely the set of befouled nodes and it is a DSet.
\end{proof}

\begin{thm}\label{dset_v_blocking}
    Let $\ev{V, Q}$ be an FBAS and let $B \subset V$ be the set of befouled nodes.
    If $B$ is a DSet, $B$ is not $v$-blocking for any intact $v$.
\end{thm}

\begin{proof}
    By Definition~\ref{def_intact_befouled}, a node $v \in V$ is intact if and only if $v \notin B$.
    By Theorem~\ref{quorum_availability_v_blocking}, $\ev{V, Q}$ enjoys quorum availability despite $B$ if and only if $B$ is not $v$-blocking for any $v \in V \setminus B$.
    Since $B$ is a DSet, $\ev{V, Q}$ enjoys quorum availability despite $B$.
    Thus $B$ is not $v$-blocking for any intact $v$.
\end{proof}
\section{Voting, Accepting, and Ratifying}

\begin{defn}\label{def_vote}
    A node $v$ votes for a statement $a$ if and only if $v$ asserts
    \begin{itemize}
        \item
            $a$ is valid,
        \item
            $a$ is consistent with all statements $v$ has accepted,
        \item
            $v$ has never voted against $a$,
        \item
            $v$ promises never to vote against $a$ in the future.
    \end{itemize}
\end{defn}

\begin{defn}\label{def_accept}
    Let $\ev{V, Q}$ be an FBAS\@, and let $v \in V$.
    $v$ accepts a statement $a$ if and only if 
    \begin{itemize}
        \item
            It has never accepted a statement contradicting $a$.
        \item
            It determines that either
            \begin{itemize}
                \item
                    There exists a quorum such that $v \in U$ and each member of $U$ either voted for $a$ or broadcast that it has accepted $a$, or
                \item
                    There exists a $v$-blocking set $B$ such that every member of $B$ broadcast that it has accepted $a$.
            \end{itemize}
    \end{itemize}
\end{defn}

Note that it is possible for a node to accept a statement it did not vote for.
Furthermore, it is possible for a node to accept a statement after voting for a contradictory statement.

\begin{defn}
    A quorum $U_a$ ratifies a statement $a$ if and only if every member of $U_a$ votes for $a$.
    A node $v$ ratifies $a$ if and only if $v$ is a member of a quorum $U_a$ that ratifies $a$.
\end{defn}

\begin{thm}\label{ratify_implies_accept}
    Let $\ev{V, Q}$ be an FBAS\@.
    If a node $v \in V$ ratifies a statement $a$, then it must accept $a$.
\end{thm}

\begin{proof}
    If a node $v$ ratifies $a$, then it is a member of a quorum $U \subset V$ that ratifies $a$.
    Thus every member of $U$ votes for $a$.
    This implies that $v$ also votes for $a$.
    By Definition~\ref{def_vote}, $v$ has never accepted a statement contradicting $a$.
    By Definition~\ref{def_accept}, $v$ accepts $a$.
\end{proof}

\begin{thm}
    If an FBAS enjoys quorum intersection and contains no ill-behaved node, then two contradictory statements cannot be both ratified.
\end{thm}

\begin{proof}
    Suppose the statement is false and let $a, \bar{a}$ denote two contradictory statements ratified in such an FBAS\@.
    Let $U_a, U_{\bar{a}}$ denote quorums ratifying such statements, respectively.
    By the definition of quorum intersection, $U_a \cap U_{\bar{a}} \ne \emptyset$.
    Let $v \in U_a \cap U_{\bar{a}}$.
    This implies that $v$ voted for both $a$ and $\bar{a}$.
    However, this goes against the definition of voting.
    In other words, $v$ must be ill-behaved, which is a contradiction to our assumption.
\end{proof}

\begin{thm}\label{ill_behaved_ratify}
    Let $\ev{V, Q}$ be an FBAS\@.
    Let $B \subsetneq V$ be a subset containing all the ill-behaved nodes and suppose that $\ev{V, Q}^B$ enjoys quorum intersection.
    Let $v_1 \ne v_2 \in V \setminus B$.
    If $v_1$ ratifies a statement $a$, then $v_2$ cannot ratify any statement that contradicts $a$.
\end{thm}

\begin{proof}
    Suppose that the theorem is false and let $U_1, U_2$ be quorums of $v_1, v_2$ that ratify $a, \bar{a}$, respectively where $a$ and $\bar{a}$ are contradictory.
    Since $v_1 \in U_1 \setminus B$,  $U_1 \setminus B \ne \emptyset$.
    By Theorem~\ref{quorum_intersection_projected_system}, $U_1' = U_1 \setminus B$ is a quorum in $\ev{V, Q}^B$.
    Similarly, $U_2' = U_2 \setminus B$ is a quorum in $\ev{V, Q}^B$.
    Since $\ev{V, Q}^B$ enjoys quorum intersection, $U_1' \cap U_2' \ne \emptyset$.
    Let $v \in U_1' \cap U_2'$.
    Then $v \in U_1 \cap U_2$.
    In order for $U_1, U_2$ to ratify $a, \bar{a}$, respectively, $v$ must vote for both $a$ and $\bar{a}$.
    However, this is against the definition of voting.
    $v$ must be an ill-behaved node, so $v \in B$, which is a contradiction because $v \in U_1 \setminus B$.
\end{proof}

\begin{thm}\label{intact_ratify_contradictory}
    Let $\ev{V, Q}$ be an FBAS with quorum intersection.
    Then two intact nodes in $V$ cannot ratify contradictory statements.
\end{thm}

\begin{proof}
    Let $v \ne v'$ be two intact nodes in $V$.
    Let $B \subset V$ be the set of befouled nodes.
    Then $v \notin B$ and $v' \notin B$.
    Since $\ev{V, Q}$ is an FBAS with quorum intersection, $B$ is a DSet by Theorem~\ref{befouled_dset}.
    By the definition of a DSet (Definition~\ref{def_dset}), $\ev{V, Q}^B$ enjoys quorum intersection.
    By Theorem~\ref{ill_behaved_ratify}, $v, v'$ cannot ratify contradictory statements.
\end{proof}

\begin{lem}\label{lem_intact_ratify}
    Let $\ev{V, Q}$ be an FBAS enjoying quorum intersection and $B$ be the set of befouled nodes.
    If $a$ is accepted by an intact node in $V$, then $a$ is ratified by some intact node in $\ev{V, Q}^B$.
\end{lem}

\begin{proof}
    Since $V$ is finite, there has to be an intact node $v$ such that no intact nodes in $V$ accepted $a$ before $v$.

    Since $\ev{V, Q}$ enjoys quorum intersection, $B$ is a DSet by Theorem~\ref{befouled_dset}.
    By Theorem~\ref{dset_v_blocking}, $B$ is not $v$-blocking.
    Therefore, by Definition~\ref{def_accept}, there must exist a quorum $U$ of $v$ such that, before $v$ accepted $a$, every member of $U$ either voted for $a$ or broadcast that it has accepted $a$.
    Because of the way we picked $v$, every intact node in $U$ must have voted for $a$ before $v$ accepted $a$.
    In other words, every node in $U \setminus B$ voted for $a$.
    By Theorem~\ref{quorum_intersection_projected_system}, $v$ ratified $a$ in ${\ev{V, Q}}^B$.
    Finally, $v$ is indeed an intact node in ${\ev{V, Q}}^B$ because ${\ev{V, Q}}^B$ contains no ill-behaved nodes.
\end{proof}

\begin{thm}\label{intact_accept_contradictory}
    Two intact nodes in an FBAS $\ev{V, Q}$ enjoying quorum intersection cannot accept contradictory statements.
\end{thm}

Note that Theorem~\ref{intact_accept_contradictory} is a stronger version of Theorem~\ref{intact_ratify_contradictory} by Theorem~\ref{ratify_implies_accept}.

\begin{proof}
    Suppose otherwise.
    Let $a, \bar{a}$ be contradictory statements accepted by two intact nodes in $\ev{V, Q}$.
    By Lemma~\ref{lem_intact_ratify}, $a, \bar{a}$ are ratified by some intact nodes in $\ev{V, Q}^B$.
    By Definition~\ref{def_dset}, $\ev{V, Q}$ enjoys quorum intersection despite $B$.
    In other words, $\ev{V, Q}^B$ enjoys quorum intersection.
    By Theorem~\ref{intact_ratify_contradictory}, $a, \bar{a}$ cannot be ratified by $v, v'$ in $\ev{V, Q}^B$, which is a contradiction.
\end{proof}

\section{Confirmation}

\begin{defn}\label{def_irrefutable}
    A statement $a$ is irrefutable in an FBAS if no intact node can ever vote against it.
\end{defn}

\begin{defn}\label{def_confirm}
    A quorum $U_a$ in an FBAS confirms a statement $a$ if and only if every element in $U_a$ broadcasts that it has accepted $a$.
    A node confirms $a$ if and only if it is in such a quorum.
\end{defn}

\begin{lem}\label{lemma_for_every_intact_node}
    Let $\ev{V, Q}$ be an FBAS with quorum intersection.
    Let $B$ denote the set of befouled nodes.
    Let $U$ be a quorum containing an intact node.
    Let $S$ be a set such that $U \subset S \subset V$.
    Let $S^{+} = S \setminus B$ be the set of intact nodes in $S$, and let $S^{-} = (V \setminus S) \setminus B$ be the set of intact nodes not in $S$.
    Either $S^{-} = \emptyset$, or $\exists v \in S^{-}$ such that $S^{+}$ is $v$-blocking.
\end{lem}

\begin{proof}
    If $\exists v \in S^{-}$ such that $S^{+}$ is $v$-blocking, then we are done.
    Suppose that $\forall v \in S^{-}$, $S^{+}$ is not $v$-blocking in $\ev{V, Q}$.
    By Theorem~\ref{v_blocking_delete}, $S^{+}$ is not $v$-blocking in $\ev{V, Q}^B$ for any $v \in S^{-} = (V \setminus B) \setminus S^{+}$.
    By Theorem~\ref{quorum_availability_v_blocking}, $\ev{V, Q}^B$ enjoys quorum availability despite $S^{+}$.
    By Definition~\ref{def_quorum_availability}, $(V \setminus B) \setminus S^{+}$ is a quorum in $\ev{V, Q}^B$, or $V \setminus B = S^{+}$.
    If $V \setminus B = S^{+}$, then $S^{-} = \emptyset$, and we are done.
    Suppose $(V \setminus B) \setminus S^{+}$ is a quorum in $\ev{V, Q}^B$.
    \begin{itemize}
        \item
            $U \setminus B$ is a quorum in $\ev{V, Q}^B$ by Theorem~\ref{quorum_intersection_projected_system}.
        \item
            Since $B$ is a DSet by Theorem~\ref{befouled_dset}, $\ev{V, Q}^B$ enjoys quorum intersection by Definition~\ref{def_dset}.
        \item
            However,
            \begin{align*}
                (U \setminus B) \cap ((V \setminus B) \setminus S^{+})
                    &= (U \setminus B) \cap S^{-} \\
                    &\subset S \cap S^{-} \\
                    &= \emptyset.
            \end{align*}
    \end{itemize}
    This is a contradiction.
\end{proof}

\begin{thm}
    If an intact node in an FBAS $\ev{V, Q}$ with quorum intersection confirms a statement $a$, then every intact node will accept and confirm $a$ once sufficient messages are delivered.
\end{thm}

\begin{proof}
    Let $B$ denote the set of befouled nodes.
    Then there exists a quorum $U \not\subset B$ such that every node in $U$ broadcast that it accepted $a$.
    After every node in $U$ broadcast it accepted $a$, there may be a node $v$ that accept $a$ since $U$ is $v$-blocking.
    After all such nodes broadcast that they accepted $a$, there may be other nodes that accept $a$ as well.
    Since $V$ is a finite set, there is a point in time where the number of nodes that accept $a$ does not increase.
    Let $S$ be the set of all nodes that accepted $a$ and broadcast it.
    \begin{itemize}
        \item
            $U$ is a quorum containing an intact node.
        \item
            $U \subset S \subset V$.
        \item
            Let $S^{+} = S \setminus B$ be the set of intact nodes in $S$, and let $S^{-} = (V \setminus S) \setminus B$ be the set of intact nodes not in $S$.
    \end{itemize}
    By Lemma~\ref{lemma_for_every_intact_node}, $S^{-}$ is empty, or $S^{+}$ is $v$-blocking for some $v \in S^{-}$.
    However, the latter is impossible because it would imply that $v$ would accept $a$.
    Therefore, $S^{-}$ is empty, and thus every intact node accepted $a$.
\end{proof}

\section{Nomination}

Nomination is done through voting, accpeting, and confirming a special type of statement in the form of \textit{nominate $x$}.

\begin{defn}
    A node $v$ is said to nominate a value $x$ if and only if it votes for the statement \textit{nominate $x$}.
\end{defn}

\begin{defn}
    A node $v$ considers a value $x$ to be a candidate if and only if $v$ has confirmed the statement \textit{nominate $x$}.
    Alternatively, we say that a node $v$ has a candidate value $x$.
\end{defn}

\begin{defn}
\end{defn}

\end{document}

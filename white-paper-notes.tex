\documentclass[12pt, psamsfonts]{amsart}

%-------Packages---------
\usepackage{amssymb,amsfonts}
\usepackage{semantic}
\usepackage{fullpage}
\usepackage{tikz-cd}
\usepackage{todonotes}
\usepackage{physics}
\usepackage[all,arc]{xy}
\usepackage{enumerate}
\usepackage{enumitem}
\usepackage{mathrsfs}
\usepackage{theoremref}
\usepackage{graphicx}
\usepackage[bookmarks]{hyperref}

%--------Theorem Environments--------
%theoremstyle{plain} --- default
\newtheorem{thm}{Theorem}[section]
\newtheorem{cor}[thm]{Corollary}
\newtheorem{prop}[thm]{Proposition}
\newtheorem{lem}[thm]{Lemma}
\newtheorem{conj}[thm]{Conjecture}
\newtheorem{quest}[thm]{Question}

\theoremstyle{definition}
\newtheorem{defn}[thm]{Definition}
\newtheorem{defns}[thm]{Definitions}
\newtheorem{con}[thm]{Construction}
\newtheorem{exmp}[thm]{Example}
\newtheorem{exmps}[thm]{Examples}
\newtheorem{notn}[thm]{Notation}
\newtheorem{notns}[thm]{Notations}
\newtheorem{addm}[thm]{Addendum}
\newtheorem*{exer}{Exercise}

\theoremstyle{remark}
\newtheorem{rem}[thm]{Remark}
\newtheorem{rems}[thm]{Remarks}
\newtheorem{warn}[thm]{Warning}
\newtheorem{sch}[thm]{Scholium}

\makeatletter
\let\c@equation\c@thm
\makeatother
\numberwithin{equation}{section}

\bibliographystyle{plain}

\begin{document}

\title{Stellar Consensus Protocol}
\author{Hidenori Shinohara}
\maketitle

\begin{abstract}
    This is my personal notes on the Stellar consensus protocol.
    This roughly follows the structure of the white paper on https://www.stellar.org/.
\end{abstract}

\begin{thm}
    If an FBAS enjoys quorum intersection and contains no ill-behaved node, then two contradictory statements cannot be both ratified.
\end{thm}

\begin{proof}
    Suppose the statement is false and let $a, \bar{a}$ denote two contradictory statements ratified in such an FBAS.
    Let $U_a, U_{\bar{a}}$ denote quorums ratifying such statements, respectively.
    By the definition of quorum intersection, $U_a \cap U_{\bar{a}} \ne \emptyset$.
    Let $v \in U_a \cap U_{\bar{a}}$.
    This implies that $v$ voted for both $a$ and $\bar{a}$.
    However, this goes against the definition of voting.
    In other words, $v$ must be ill-behaved, which is a contradiction to our assumption.
\end{proof}


\end{document}



\subsection{Ballot Protocol}

\begin{defn}[Ballot]
    A ballot $b$ is a pair of the form $b = \ev{n, x}$ where $x \ne \bot$ is a value and $n \geq 1$ is a counter.
\end{defn}

\begin{defn}[Order]
    For any ballots $b_1, b_2$, $b_1 \leq b_2$ if and only if $(b_1.n, b_1.x) \leq (b_2.n, b_2.x)$ in dictionary order.
\end{defn}

\begin{defn}[Null ballot]
    $\textbf{0} = \ev{0, \bot}$ is a special null invalid ballot that is less than any other ballots.
\end{defn}

\begin{defn}[Compatability]
    For a pair of ballots $b_1, b_2$, we define $\sim, \nsim, \lesssim, \lnsim$ as following:
    \begin{align*}
        b_1 \sim b_2 &\iff b_1.x = b_2.x \\
        b_1 \nsim b_2 &\iff b_1.x \ne b_2.x \\
        b_1 \lesssim b_2 &\iff b_1 \leq b_2 \land b_1 \sim b_2 \\
        b_1 \lnsim b_2 &\iff b_1 \leq b_2 \land b_1 \nsim b_2.
    \end{align*}
\end{defn}

\begin{defn}[Commit]
    For a given ballot $b$, \textit{commit} $b$ is a shorthand for ``I have confirmed \textoverline{\textit{commit} $b'$} for all $b' \lnsim b$."
\end{defn}

\textoverline{\textit{commit} $b'$} denotes a statement that contradicts \textit{commit} $b'$.

\begin{rem}[Abort]
    For a given ballot $b$, we will often write \textit{abort} $b$ instead of \textoverline{\textit{commit} $b$} for readability.
\end{rem}

\begin{exmp}
    Let $a, b, c, d, e$ be statements with $a < b < c < d < e$.
    The following table shows what statements a node needs to confirm to abort before voting for \textit{commit} $(3, c)$:
    \begin{center}
        \begin{tabular}{|c|c|c|c|c|c|}
            \hline
            $(n, x)$ & $x = a$                           & $x = b$                           & $x = c$ & $x = d$                           & $x = e$                           \\ \hline
            $n = 1$  & \cellcolor{blue!25}\textit{abort} & \cellcolor{blue!25}\textit{abort} &         & \cellcolor{blue!25}\textit{abort} & \cellcolor{blue!25}\textit{abort} \\ \hline
            $n = 2$  & \cellcolor{blue!25}\textit{abort} & \cellcolor{blue!25}\textit{abort} &         & \cellcolor{blue!25}\textit{abort} & \cellcolor{blue!25}\textit{abort} \\ \hline
            $n = 3$  & \cellcolor{blue!25}\textit{abort} & \cellcolor{blue!25}\textit{abort} &         &                                   &                                   \\ \hline
            $n = 4$  &                                   &                                   &         &                                   &                                   \\ \hline
        \end{tabular}
    \end{center}

    There are a few things that may be worth mentioning:
    \begin{itemize}
        \item
            $(2, c)$ is \textit{not} $\lnsim (3, c)$ because $(2, c) \sim (3, c)$.
        \item
            $(3, a) \lnsim (3, c)$ because $(3, a) \leq (3, c)$ and $(3, a) \nsim (3, c)$.
        \item
            $(3, d)$ is \textit{not} $\lnsim (3, c)$ even though $(3, d) \nsim (3, c)$.
            This is because $(3, d)$ is \textit{not} $\leq (3, c)$.
    \end{itemize}
\end{exmp}

\begin{defn}[Prepare]\label{def_prepare}
    Let $b$ be a ballot.
    A node is said to vote, accept, or confirm that \textit{$b$ is prepared} if and only if it votes, accepts, or confirms \textit{abort} $b'$ for all $b' \lnsim b$, respectively.
\end{defn}

\begin{rem}
    By the definition of committing and preparing, a node can vote for \textit{commit} $b$ if and only if it has confirmed that $b$ is prepared.
\end{rem}

\begin{rem}
    The approach presented here is slightly different from the white paper's.
    In the white paper, the fact that \textit{commit} $b$ is valid to vote for only if $b$ is confirmed prepared is given as a (seemingly arbitrary) ``rule."
    However, I was not very convinced by this approach because it would mean that we need to alter the definition of voting.
    More specifically, the definition of voting states that one votes for a statement \textit{if and only if} the four conditions have been met.  
    However, this new ``rule" may prevent a node from voting even when all of the four conditions.

    In the approach presented above, the ``rule" regarding voting for a \textit{commit} $b$ is embedded in the commit statement itself.
    Therefore, we can safely apply all of our FBAS arguments.
\end{rem}

\begin{defn}[Externalize]
    A node \textit{externalizes} a value $x$ if and only if it confirms \textit{commit} $\ev{n, x}$ for some $n \geq 1$.
\end{defn}

\begin{thm}
    Let $b = (b.n, b.x)$ be given.
    In an FBAS with a quorum intersection, if an intact node confirms that $b$ is prepared, then every intact node will eventually confirm that $b$ is prepared.
\end{thm}

\begin{proof}
    By Definition~\ref{def_prepare}, confirming that $b$ is prepared is equivalent to confirming \textit{abort} $b'$ for all  $b' \lnsim  b$.
    For each $b' \lnsim b$, Theorem~\ref{all_intact_nodes_accept_confirm_same_things} states that all intact nodes will eventually confirm \textit{abort} $b'$.
    Therefore, all intact nodes will eventually confirm that $b$ is prepared.
\end{proof}

\begin{thm}
    Let $\ev{V, Q}$ be an FBAS with quorum intersection.
    Let $v \in V$ be an intact node.
    If $v$ accepts \textit{commit} $b$, all intact nodes will eventually confirm \textit{commit} $b'$ for some $b' \sim b$.
\end{thm}

\begin{proof}
    Let $w$ be an intact node and suppose $w$ accepts \textit{commit} $b'$ for some ballot $b'$.
    If $b \sim b'$, we are done.
    Suppose otherwise.

    Since $v, w$ accept $b, b'$, they must have confirmed that $b, b'$ are prepared, respectively.
    Without loss of generality, $b \lnsim b'$.
    By Definition~\ref{def_prepare}, $w$ must have confirmed \textit{abort} $b$.
    \textit{abort} $b$ and \textit{commit} $b$ are defined to be contradictory, and Theorem~\ref{intact_accept_contradictory} states that no intact node can accept contradictory statements.
    Therefore, this is a contradiction, so $b \sim b'$.

    Since a node must first accept a statement before confirming it, the argument above shows that if an intact node confirms \textit{commit} $b'$, then $b' \sim b$.
    
    Now, we need to show that all intact nodes will eventually confirm \textit{commit} $b'$ for some $b' \sim b$.
    \todo[inline,caption={}]{
        This ``liveness" property requires the concrete ballot protocol.
    }
\end{proof}

\subsubsection{Concrete Ballot Protocol}


\begin{defn}\label{Ballot State}
    The tuple $(\varphi, b, p', p, c, h, z, M)$ where
    \begin{itemize}
        \item
            $\varphi \in \{\text{PREPARE, CONFIRM, EXTERNALIZE}\}$,
        \item
            $b, p', p, c, h$ are ballots,
        \item
            $z$ is a value,
        \item
            $M$ is a set of messages
    \end{itemize}
    is called a ballot state, and each node must maintain one during the ballot protocol.
\end{defn}


\begin{defn}\label{Ballot Message}
    Any message that a well-behaved node sends is one of the following messages:
    \begin{itemize}
        \item
            $(\text{PREPARE}, v, i, b, p, p', c.n, h.n, D)$,
        \item
            $(\text{CONFIRM}, v, i, b, p.n, p'.n, c.n, h.n, D)$,
        \item
            $(\text{EXTERNALIZE}, v, i, c.x, c.n, h.n, D)$,
    \end{itemize}
    and the phase in the message must match the phase in the state.
\end{defn}

\begin{defn}\label{Initial Ballot State}
    The state such that
    \begin{itemize}
        \item
            $\varphi = \text{PREPARE}$
        \item
            $b = p = p' = c = h = \textbf{0}$
        \item
            $z = \bot$
        \item
            $M = \emptyset$
    \end{itemize}
    is called the initial ballot state, and every well-behaved node must have this state when the ballot protocol starts.
\end{defn}


There are exactly two ways in which a well-behaved node is allowed to change the state:
\begin{itemize}
    \item
        When a node receives a new nomination message.
    \item
        When a node receives a new ballot message.
\end{itemize}

We will call them Transition Type 1 and Type 2, respectively.

\begin{defn}[Transition Type 1]
    When $c_{high} = 0$, a node may update $x_{next}$ in response to a NOMINATE message.
\end{defn}

\begin{defn}[Transition Type 2]
    \todo[inline,caption={}]{
        proofread this
    }
    Upon receiving a new ballot message $m$, node $v$ adds $m$ to $M$ and updates its state as following:
    \begin{enumerate}
        \item
            If $\varphi = \text{PREPARE}$ and $m$ lets $v$ accept new ballots as prepared, update $p$ and $p'$.
            Afterwards, if either $p \gnsim h$ or $p' \gnsim h$, then set $c \leftarrow \textbf{0}$.
        \item
            If $\varphi = \text{PREPARE}$ and $m$ lets $v$ confirm new higher ballots prepared, then raise $h$ to the highest such ballot and set $z \leftarrow h.x$.
        \item
            If $\varphi = \text{PREPARE}$ and $v$ accepts \textit{commit} for one or more ballots, set $c$ to the lowest such ballot, then set $h$ to the highest ballot such that $v$ accepts all $\{ \textit{commit } b' \mid c \lesssim b' \lesssim h \}$, and set $\varphi \leftarrow \text{CONFIRM}$.
            Also, set $z \leftarrow h.x$ after updating $h$, and unless $h \lesssim b$, set $b \leftarrow h$.
        \item
            If $\varphi = \text{CONFIRM}$ and the received messaged lets $v$ accept new ballots prepared, raise $p$ to the highest accepted prepared ballot such that $p \sim c$.
        \item
            If $\varphi = \text{CONFIRM}$ and the received message lets $v$ accept new ballots prepared, raise $P$ to the highest accepted ballot such that $p \sim c$.
        \item
            If $\varphi = \text{CONFIRM}$ and $v$ accepts more commit messages or raise $b_{cur}$, then let $h'$ be the highest ballot such that $v$ accepts all $\{ \textit{commit } b' \mid b \lnsim b' \lnsim h' \}$  (if any).
            If there exists such an $h'$ and $h' > h$, then set $h \leftarrow h'$, and, if necessary, raise $c$ to the lowest ballot such that $v$ accepts all $\{ \textit{commit } b' \mid c \lnsim b' \lnsim h \}$.
        \item
            If $\varphi = \text{CONFIRM}$ and $v$ confirms $\textit{commit } c'$ for any $c'$, set $c$ and $h$ to the lowest and highest such ballots, set $\varphi \leftarrow \text{EXTERNALIZE}$, externalize $c.x$ and terminate.
        \item
            If $\varphi = \{ \text{PREPARE}, \text{CONFIRM} \}$ and $\exists S \subseteq M_v$ such that the set of senders $\{ v_{m'} \mid m' \in S \}$ is $v$-blocking and $\forall m' \in S, b_{m'}.n > b_v.n$, then set $b \leftarrow \ev{n, z}$, where $n$ is the lowest coutner for which no such $S$ exists.
            Repeat the previous steps after updating $b$.
    \end{enumerate}
\end{defn}

The 9 steps to update the local state that appears on P.24 can be understood as following:

% \begin{enumerate}
%     \item
%         If $\varphi = \text{PREPARE}$ and $m$ lets $v$ accept new ballots as prepared, update $p$ and $p'$.
%         Afterwards, if either $p \gnsim h$ or $p' \gnsim h$, then set $c \leftarrow \textbf{0}$.
%         \begin{itemize}
%             \item
%                 $p, p'$ denote the two highest ballots accepted as prepared.
%                 Thus it makes sense to update $p$ and $p'$ when $v$ accepts new ballots.
%                 \todo[inline,caption={}]{
%                     TODO: Why do we need to set $c$ to $\textbf{0}$?
%                 }
%         \end{itemize}
%     \item
%         If $\varphi = \text{PREPARE}$ and $m$ lets $v$ confirm new higher ballots prepared, then raise $h$ to the highest such ballot and set $z \leftarrow h.x$.
%         \begin{itemize}
%             \item
%                 $h$ denotes the highest ballot confirmed as prepared, so it makes sense to update $h$ in this case.
%                 Figure 16 of the white paper mentions that $z = h.x$ as long as $h \ne \textbf{0}$, so it makes sense to update $z$ as well.
%         \end{itemize}
%     \item
%         If $\varphi = \text{PREPARE}, c = \texbf{0}, b \leq h$, and neither $p \gnsim h$ nor $p' \gnsim h$, then set $c$ to the lowest ballot satisfying $b \leq c \lnsim h$.
%         \begin{itemize}
%             \item
%                 \todo[inline,caption={}]{
%                     TODO: What does this do? I don't really understand this.
%                     What does $b \leq h, p \gnsim h, p' \gnsim$ mean?
%                 }
%         \end{itemize}
%     \item
%         If $\varphi = \text{PREPARE}$ and $v$ accepts \textit{commit} for one or more ballots, set $c$ to the lowest such ballot, then set $h$ to the highest ballot such that $v$ accepts all $\{ \textit{commit } b' \mid c \lesssim b' \lesssim h \}$, and set $\varphi \leftarrow \text{CONFIRM}$.
%         Also, set $z \leftarrow h.x$ after updating $h$, and unless $h \lesssim b$, set $b \leftarrow h$.
%         \begin{itemize}
%             \item
%                 One might wonder how $v$ can accept $\textit{commit}$ for more than one ballot at the same time.
%                 This is possible only if they all have the same value, but different counters.
%             \item
%                 \todo[inline,caption={}]{
%                     TODO: Why don't we just set $h$ to the highest such ballot?
%                 }
%             \item
%                 Setting $s \leftarrow h.x$ makes sense since if $h \ne \textbf{0}$, $z = h.x$ as mentioned in Figure 16 of the white paper.
%             \item
%                 \todo[inline,caption={}]{
%                     Is it possible that $h \lesssim b$ and $h \ne b$?
%                 }
%             \item
%                 Setting $b \leftarrow h$ means $v$ will try to confirm $\textit{commit } h$.
%         \end{itemize}
%     \item
%         If $\varphi = \text{CONFIRM}$ and the received messaged lets $v$ accept new ballots prepared, raise $p$ to the highest accepted prepared ballot such that $p \sim c$.
%         \begin{itemize}
%             \item
%                 Since $\varphi = \text{CONFIRM}$, $v$ must have accepted at least one $\textit{commit}$ statement.
%                 Therefore, $v$ has already decided on which value to confirm a $\textit{commit}$ statemnt for, if any.
%                 Thus all $v$ can do is to continue accepting $\texit{commit}$ statements with the same value.
%                 \todo[inline,caption={}]{
%                     Okay, this is true, but I'm not sure how this will help me understand this step.
%                 }
%         \end{itemize}
% \end{enumerate}
% 

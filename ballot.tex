\subsection{Ballot Protocol}

\begin{defn}[Ballot]
    A ballot $b$ is a pair of the form $b = \ev{n, x}$ where $x \ne \bot$ is a value and $n \geq 1$ is a counter.
\end{defn}

\begin{defn}[Order]
    For any ballots $b_1, b_2$, $b_1 \leq b_2$ if and only if $(b_1.n, b_1.x) \leq (b_2.n, b_2.x)$ in dictionary order.
\end{defn}

\begin{defn}[Null ballot]
    $\textbf{0} = \ev{0, \bot}$ is a special null invalid ballot that is less than any other ballots.
\end{defn}

\begin{defn}[Compatability]
    For a pair of ballots $b_1, b_2$, we define $\sim, \nsim, \lesssim, \lnsim$ as following:
    \begin{align*}
        b_1 \sim b_2 &\iff b_1.x = b_2.x \\
        b_1 \nsim b_2 &\iff b_1.x \ne b_2.x \\
        b_1 \lesssim b_2 &\iff b_1 \leq b_2 \land b_1 \sim b_2 \\
        b_1 \lnsim b_2 &\iff b_1 \leq b_2 \land b_1 \nsim b_2.
    \end{align*}
\end{defn}

\begin{defn}[Commit]
    For a given ballot $b$, \textit{commit} $b$ is a shorthand for ``I have confirmed \textoverline{\textit{commit} $b'$} for all $b' \lnsim b$."
\end{defn}

\textoverline{\textit{commit} $b'$} denotes a statement that contradicts \textit{commit} $b'$.

\begin{rem}[Abort]
    For a given ballot $b$, we will often write \textit{abort} $b$ instead of \textoverline{\textit{commit} $b$} for readability.
\end{rem}

\begin{exmp}
    Let $a, b, c, d, e$ be statements with $a < b < c < d < e$.
    The following table shows what statements a node needs to confirm to abort before voting for \textit{commit} $(3, c)$:
    \begin{center}
        \begin{tabular}{|c|c|c|c|c|c|}
            \hline
            $(n, x)$ & $x = a$                           & $x = b$                           & $x = c$ & $x = d$                           & $x = e$                           \\ \hline
            $n = 1$  & \cellcolor{blue!25}\textit{abort} & \cellcolor{blue!25}\textit{abort} &         & \cellcolor{blue!25}\textit{abort} & \cellcolor{blue!25}\textit{abort} \\ \hline
            $n = 2$  & \cellcolor{blue!25}\textit{abort} & \cellcolor{blue!25}\textit{abort} &         & \cellcolor{blue!25}\textit{abort} & \cellcolor{blue!25}\textit{abort} \\ \hline
            $n = 3$  & \cellcolor{blue!25}\textit{abort} & \cellcolor{blue!25}\textit{abort} &         &                                   &                                   \\ \hline
            $n = 4$  &                                   &                                   &         &                                   &                                   \\ \hline
        \end{tabular}
    \end{center}

    There are a few things that may be worth mentioning:
    \begin{itemize}
        \item
            $(2, c)$ is \textit{not} $\lnsim (3, c)$ because $(2, c) \sim (3, c)$.
        \item
            $(3, a) \lnsim (3, c)$ because $(3, a) \leq (3, c)$ and $(3, a) \nsim (3, c)$.
        \item
            $(3, d)$ is \textit{not} $\lnsim (3, c)$ even though $(3, d) \nsim (3, c)$.
            This is because $(3, d)$ is \textit{not} $\leq (3, c)$.
    \end{itemize}
\end{exmp}

\begin{defn}[Prepare]\label{def_prepare}
    Let $b$ be a ballot.
    A node is said to vote, accept, or confirm that \textit{$b$ is prepared} if and only if it votes, accepts, or confirms \textit{abort} $b'$ for all $b' \lnsim b$, respectively.
\end{defn}

\begin{rem}
    By the definition of committing and preparing, a node can vote for \textit{commit} $b$ if and only if it has confirmed that $b$ is prepared.
\end{rem}

\begin{rem}
    The approach presented here is slightly different from the white paper's.
    In the white paper, the fact that \textit{commit} $b$ is valid to vote for only if $b$ is confirmed prepared is given as a (seemingly arbitrary) ``rule."
    However, I was not very convinced by this approach because it would mean that we need to alter the definition of voting.
    More specifically, the definition of voting states that one votes for a statement \textit{if and only if} the four conditions have been met.  
    However, this new ``rule" may prevent a node from voting even when all of the four conditions.

    In the approach presented above, the ``rule" regarding voting for a \textit{commit} $b$ is embedded in the commit statement itself.
    Therefore, we can safely apply all of our FBAS arguments.
\end{rem}

\begin{defn}[Externalize]
    A node \textit{externalizes} a value $x$ if and only if it confirms \textit{commit} $\ev{n, x}$ for some $n \geq 1$.
\end{defn}

\begin{thm}
    Let $b = (b.n, b.x)$ be given.
    In an FBAS with a quorum intersection, if an intact node confirms that $b$ is prepared, then every intact node will eventually confirm that $b$ is prepared.
\end{thm}

\begin{proof}
    By Definition~\ref{def_prepare}, confirming that $b$ is prepared is equivalent to confirming \textit{abort} $b'$ for all  $b' \lnsim  b$.
    For each $b' \lnsim b$, Theorem~\ref{all_intact_nodes_accept_confirm_same_things} states that all intact nodes will eventually confirm \textit{abort} $b'$.
    Therefore, all intact nodes will eventually confirm that $b$ is prepared.
\end{proof}

\begin{thm}
    Let $\ev{V, Q}$ be an FBAS with quorum intersection.
    Let $v \in V$ be an intact node.
    If $v$ accepts \textit{commit} $b$, all intact nodes will eventually confirm \textit{commit} $b'$ for some $b' \sim b$.
\end{thm}

\begin{proof}
    Let $w$ be an intact node and suppose $w$ accepts \textit{commit} $b'$ for some ballot $b'$.
    If $b \sim b'$, we are done.
    Suppose otherwise.

    Since $v, w$ accept $b, b'$, they must have confirmed that $b, b'$ are prepared, respectively.
    Without loss of generality, $b \lnsim b'$.
    By Definition~\ref{def_prepare}, $w$ must have confirmed \textit{abort} $b$.
    \textit{abort} $b$ and \textit{commit} $b$ are defined to be contradictory, and Theorem~\ref{intact_accept_contradictory} states that no intact node can accept contradictory statements.
    Therefore, this is a contradiction, so $b \sim b'$.

    Since a node must first accept a statement before confirming it, the argument above shows that if an intact node confirms \textit{commit} $b'$, then $b' \sim b$.
    
    Now, we need to show that all intact nodes will eventually confirm \textit{commit} $b'$ for some $b' \sim b$.
    \todo[inline,caption={}]{
        This ``liveness" property requires the concrete ballot protocol.
    }
\end{proof}

\subsubsection{Concrete Ballot Protocol}

\begin{defn}[Ballot State]
    The tuple $(\varphi, b, p', p, c, h, z, M)$ where
    \begin{itemize}
        \item
            $\varphi \in \{\text{PREPARE, CONFIRM, EXTERNALIZE}\}$,
        \item
            $b, p', p, c, h$ are ballots,
        \item
            $z$ is a value,
        \item
            $M$ is a set of messages
    \end{itemize}
    is called a ballot state, and each node must maintain one during the ballot protocol.
\end{defn}

\begin{defn}[Meaning of Variables]
    $ $
    \todo[inline,caption={}]{
        FInish the rest.
    }
    \begin{itemize}
        \item
            $p', p$ are the two highest ballots accepted as prepared such that $p' \lnsim p$, where $p' = \textbf{0}$ or $p = p' = \textbf{0}$ if there are no such ballots.
    \end{itemize}
\end{defn}

\begin{defn}[Ballot Message]
    Any message that a well-behaved node sends is one of the following messages:
    \begin{itemize}
        \item
            $(\text{PREPARE}, v, i, b, p, p', c.n, h.n, D)$,
        \item
            $(\text{CONFIRM}, v, i, b, p.n, p'.n, c.n, h.n, D)$,
        \item
            $(\text{EXTERNALIZE}, v, i, c.x, c.n, h.n, D)$,
    \end{itemize}
    and the phase in the message must match the phase in the state.
\end{defn}

\begin{defn}[PREPARE]
    $(\text{PREPARE}, v, i, b, p, p', c.n, h.n, D)$ encodes
    \begin{itemize}
        \item
            $\{ \textit{abort } b' \lor \textit{accept(abort $b'$)} \mid b' \lnsim b \}$
        \item
            $\{ \textit{accept(abort $b'$)} \mid b' \lnsim p \}$
        \item
            $\{ \textit{accept(abort $b'$)} \mid b' \lnsim p' \}$
        \item
            $\{ \textit{commit } b' \mid c.n \ne 0 \land c \lesssim b' \lesssim h \}$
    \end{itemize}
\end{defn}

\todo[inline,caption={}]{
    Add CONFIRM, EXTERNALIZE.
    But when I think about it, this part may not be super necessary?
    Can we prove all the invariants without defining what messages look like?
}

\begin{defn}\label{Initial Ballot State}
    The state such that
    \begin{itemize}
        \item
            $\varphi = \text{PREPARE}$
        \item
            $b = p = p' = c = h = \textbf{0}$
        \item
            $z = \bot$
        \item
            $M = \emptyset$
    \end{itemize}
    is called the initial ballot state, and every well-behaved node must have this state when the ballot protocol starts.
\end{defn}

There are exactly two ways in which a well-behaved node is allowed to change the state:
\begin{itemize}
    \item
        When a node receives a new nomination message.
    \item
        When a node receives a new ballot message.
\end{itemize}

We will call them Transition Type 1 and Type 2, respectively.

\begin{defn}[Transition Type 1]\label{transition_type_1}
    When $h = \textbf{0}$, a node may update $z$ in response to a NOMINATE message.
    If $z$ was $\bot$, then $b \leftarrow \ev{1, z}$.
\end{defn}

\begin{defn}[Transition Type 2]\label{transition_type_2}
    Upon receiving a new ballot message $m$, node $v$ adds $m$ to $M$ and updates its state as following:
    \begin{enumerate}
        \item % 1
            If $\varphi = \text{PREPARE}$ and $m$ lets $v$ accept new ballots as prepared, update $p$ and $p'$.
            Afterwards, if either $p \gnsim h$ or $p' \gnsim h$, then set $c \leftarrow \textbf{0}$.
        \item % 2
            If $\varphi = \text{PREPARE}$ and $m$ lets $v$ confirm new higher ballots prepared, then raise $h$ to the highest such ballot and set $z \leftarrow h.x$.
        \item % 3
            If $\varphi = \text{PREPARE}, c = \textbf{0}, b \leq h$, and neither $p \gnsim h$ nor $p' \gnsim h$, then set $c$ to the lowest ballot satisfiying $b \leq c \lesssim h$.
        \item % 4
            If $\varphi = \text{PREPARE}$ and $v$ accepts \textit{commit} for one or more ballots, set $c$ to the lowest such ballot, then set $h$ to the highest ballot such that $v$ accepts all $\{ \textit{commit } b' \mid c \lesssim b' \lesssim h \}$, and set $\varphi \leftarrow \text{CONFIRM}$.
            Also, set $z \leftarrow h.x$ after updating $h$, and unless $h \lesssim b$, set $b \leftarrow h$.
        \item % 5
            If $\varphi = \text{CONFIRM}$ and the received message lets $v$ accept new ballots prepared, raise $p$ to the highest accepted prepared ballot such that $p \sim c$.
        \item % 6
            If $\varphi = \text{CONFIRM}$ and $v$ accepts more commit messages or raises $b$, then let $h'$ be the highest ballot such that $v$ accepts all $\{ \textit{commit } b' \mid b \lesssim b' \lesssim h' \}$  (if any).
            If there exists such an $h'$ and $h' > h$, then set $h \leftarrow h'$, and, if necessary, raise $c$ to the lowest ballot such that $v$ accepts all $\{ \textit{commit } b' \mid c \lesssim b' \lesssim h \}$.
        \item % 7
            If $\varphi = \text{CONFIRM}$ and $v$ confirms $\textit{commit } c'$ for any $c'$, set $c$ and $h$ to the lowest and highest such ballots, set $\varphi \leftarrow \text{EXTERNALIZE}$, externalize $c.x$ and terminate.
        \item % 8
            If $\varphi \in \{ \text{PREPARE}, \text{CONFIRM} \}$ and $b < h$, then set $b \leftarrow h$.
        \item % 9
            If $\varphi \in \{ \text{PREPARE}, \text{CONFIRM} \}$ and $\exists S \subseteq M_v$ such that the set of senders $\{ v_{m'} \mid m' \in S \}$ is $v$-blocking and $\forall m' \in S, b_{m'}.n > b_v.n$, then set $b \leftarrow \ev{n, z}$, where $n$ is the lowest coutner for which no such $S$ exists.
            Repeat the previous steps after updating $b$.
    \end{enumerate}
\end{defn}

We will now prove invariants of states in the concrete ballot protocol.

\begin{lem}\label{nonzero_h_if_confirmed}
    If $v$ has confirmed any ballot as prepared, $h \ne \textbf{0}$.
\end{lem}

\begin{proof}
    It suffices to show that it is impossible that
    \begin{enumerate}[label=(\alph*)]
        \item % a
            $v$ has confirmed a ballot as prepared, and
        \item % b
            $h = \textbf{0}$.
    \end{enumerate}

    This invariant holds in the initial state because $v$ has not confirmed anything initially.

    Suppose that the invariant holds in the current state.
    Then there are $3 = 2^2 - 1$ possible ways in which the invariant can be violated.
    \begin{enumerate}[label=(\roman*)]
        \item % i
            (a) currently holds and (b) does not.
            After some actions caused by a new message, (b) becomes true.
        \item % ii
            (a) does not hold and (b) currently holds.
            After some actions caused by a new message, (a) becomes true.
        \item % iii
            Neither (a) nor (b) holds.
            After some actions caused by a new message, both become true.
    \end{enumerate}
    Since none of the actions sets $h \leftarrow \textbf{0}$, (i) and (iii) are impossible.
    Moreover, if $h = \textbf{0}$ and a new message lets $v$ confirm a ballot as prepared, Step (2) forces $v$ to update the value of $h$ to a nonzero value.
    It is possible that $v$ performs some other actions after Step (2), but no action sets $h \leftarrow \textbf{0}$, so (b) will never hold.
    In other words, (ii) is impossible.
\end{proof}

\begin{lem}\label{lem_h_highest_ballot}
    If $\varphi = \text{PREPARE}$ and $h \ne \textbf{0}$, then $h$ is the highest ballot confirmed as prepared.
\end{lem}

\begin{proof}
    It suffices to show that, at any given moment,
    \begin{enumerate}[label=(\alph*)]
        \item % (a)
            $\varphi \ne \text{PREPARE}$, or
        \item % (b)
            $h = \textbf{0}$, or
        \item % (c)
            $h$ is the highest ballot confirmed as prepared.
    \end{enumerate}
    This invariant holds in the initial state since $h = \textbf{0}$.
    Suppose that the invariant is true in the current state.
    There are $7 = 2^3 - 1$ possible ways in which the invariant can be violated.
    \begin{enumerate}[label=(\roman*)]
        \item % i
            (a), (b), (c) all hold in the current state, but all of them become false after one message.
        \item % ii
            Only (a) and (b) hold in the current state, but they become false after one message.
        \item % iii
            Only (a) and (c) hold in the current state, but they become false after one message.
        \item % iv
            Only (b) and (c) hold in the current state, but they become false after one message.
        \item % v
            Only (a) holds in the current state, but it becomes false after one message.
        \item % vi
            Only (b) holds in the current state, but it becomes false after one message.
        \item % vii
            Only (c) holds in the current state, but it becomes false after one message.
    \end{enumerate}

    First, it is easy to see that, if $\varphi \ne \text{PREPARE}$, then no message can set $\varphi \leftarrow \text{PREPARE}$.
    Therefore, (i), (ii), (iii) and (v) are impossible.

    By taking the contrapositive of Lemma~\ref{nonzero_h_if_confirmed}, we know that if $h = \textbf{0}$, then $v$ has not confirmed any ballot as prepared.
    Thus (iv) is impossible.

    We will now examine option (vi).
    Suppose $\varphi = \text{PREPARE}, h = \textbf{0}$.
    The only way for $v$ to set $h$ to a ballot that is not $\textbf{0}$ while having $\varphi = \text{PREPARE}$ is Step (2) in Transition Type 2(\ref{transition_type_2}).
    However, Step (2) sets $h$ to the highest ballot confirmed as prepared, so (c) becomes true through the step.
    Therefore, (vi) is impossible.

    Similarly, consider option (vii).
    Then we have $\varphi = \text{PREPARE}, h \ne \textbf{0}$, and $h$ is the highest ballot confirmed as prepared.
    For (c) to become false, we need to either update the value of $h$, or confirm another ballot as prepared while keeping $\varphi = \text{PREPARE}$.
    This means we must take Step (2) in Transition Type 2(\ref{transition_type_2}), and that means we will update $h$ to the highest ballot confirmed as prepared.
    Thus (c) continues to hold and (vii) is impossible.

    Hence, we confirm that the invariant always holds.
\end{proof}

\begin{thm}\label{h_highest_ballot_confirmed_prepared}
    Suppose $\varphi = \text{PREPARE}$.
    $h$ is the highest ballot confirmed as prepared, or $\textbf{0}$ if none.
\end{thm}

\begin{proof}
    Lemma~\ref{nonzero_h_if_confirmed} and~\ref{lem_h_highest_ballot}.
\end{proof}


\begin{lem}
    If $\varphi = \text{PREPARE}$, then $v$ has confirmed $c$ as prepared.
\end{lem}

\begin{proof}
    It suffices to show that, at any given moment,
    \begin{enumerate}[label=(\alph*)]
        \item % (a)
            $\varphi \ne \text{PREPARE}$, or
        \item % (b)
            $v$ has confirmed $c$ as prepared.
    \end{enumerate}
    Note that every node confirms $\textbf{0}$ as prepared vacuously because $\{ \textit{abort } b' \mid b' \lnsim \textbf{0} \} = \emptyset$.
    Thus the invariant holds initially since (b) holds in the initial state.
    Suppose that the invariant holds in some state.
    Since no action sets $\varphi \leftarrow \text{PREPARE}$, there is no state in which (a) becomes true after an action.
    Thus we only need to consider the possibility where only (b) holds in the current state, and (b) becomes false after an action.
    In other words, $\varphi = \text{PREPARE}$ and $v$ has confirmed $c$ as prepared in the current state, but after an action, $v$ has not confirmed $c$ as prepared.
    Since the FBA does not have a way to ``un-confirm" a statement, this means $c$ changed to something $v$ has not confirmed as prepared.
    There are exactly two actions that modify $c$ while keeping $\varphi = \text{PREPARE}$:
    \begin{itemize}
        \item
            Step (1) may set $c \leftarrow \textbf{0}$.
            As discussed above, every node confirms $\textbf{0}$ as prepared vacuously, so (b) cannot become false after Step (1).
        \item
            Step (3) may update $c$ such that $c \lesssim h$.
            Lemma~\ref{lem_h_highest_ballot} shows that $v$ must have confirmed $h$ as prepared.
            In other words, $v$ must have confirmed abort $b'$ for all $b' \lnsim h$.
            Since $b' \lnsim c \land c \lesssim h \implies b' \lnsim h$, $v$ must have confirmed the new $c$ as prepared.
            In other words, (b) does not become false after Step (3).
    \end{itemize}
    Therefore, if $\varphi = \text{PREPARE}$, then $v$ has confirmed $c$ as prepared.
\end{proof}

\begin{lem}\label{c_n_leq_h_n_leq_b_n}
    $c.n \leq h.n \leq b.n$.
\end{lem}

\begin{proof}
    This holds in the initial state since $c.n = h.n = b.n = 0$.
    The only way to violate this invariant is to start with a state where only (b) holds, and after an action (b) no longer holds.
    There are several actions that modify at least one of $c.n, h.n, b.n$:
    \begin{itemize}
        \item
            Transition Type 1:
            Suppose $c.n \leq h.n \leq b.n$.
            If Transition Type 1 is applied, $h = \textbf{0}$.
            Thus $c.n = h.n = 0$.
            Transtion Type 1 may update $b \leftarrow \ev{1, z}$.
            Since $0 \leq 1$, (b) continues to hold after Transition Type 1.
        \item
            Transition Type 2:
            Step (9) requires the node to repeat the previous steps if some action takes place.
            Thus Step (8) is the last step that any node tries to apply upon receiving any new message.
            Since Step (8) ensures that $h \leq b$, we know that $h.n \leq b.n$ always.
            We will now check if $c.n \leq h.n$ continues to hold.
            \begin{itemize}
                \item
                    Step (2) modifies $h$.
                    Suppose that $v$ had confirmed at least one ballot prepared.
                    By Theorem~\ref{h_highest_ballot_confirmed_prepared}, $h$ is the highest ballot confirmed prepared.
                    Step (2) raises $h$ to a higher ballot, so $c.n \leq h.n$ continues to hold.

                    Now we will consider the case where $v$ had not confirmed any ballot prepared.
                    By Theorem~\ref{h_highest_ballot_confirmed_prepared}, $h = \textbf{0}$.
                    Step (2) raises $h$ to a higher ballot, so $c.n \leq h.n$ continues to hold.
                \item
                    Step (3), (4) and (6) explicitly set $c$ such that $c \lesssim h$, so $c.n \leq h.n$ continues to hold.
                \item
                    Step (7) explicitly require $c \leq h$, so $c.n \leq h.n$.
            \end{itemize}
    \end{itemize}
    Therefore, $c.n \leq h.n \leq b.n$.
\end{proof}

\begin{rem}
    Lemma~\ref{c_n_leq_h_n_leq_b_n} may not hold while executing some steps.
    For instance, Step (2) may update $h$ such that $b < h$, and Lemma~\ref{c_n_leq_h_n_leq_b_n} may not hold until we get to Step (8).
    Here, we only consider the state after completing all the actions associated to the new message.
\end{rem}

\begin{thm}
    Suppose $\varphi = \text{PREPARE}$.
    If $c \ne \textbf{0}$, then $c$ is the lowest ballot for which $v$ has voted \textit{commit} and not accepted \textit{abort}.
\end{thm}

\begin{proof}
    \todo[inline,caption={}]{
        Prove this!
        PREPARE messages encode a vote to \textit{commit} $c, \cdots, h$ if $c \ne \textbf{0}$.
        So, $v$ has voted $\textit{commit }c$ for sure if $c \ne \textbf{0}$.
        How do I know that it is the lowest such ballot?
        The accepted part is easy because as soon as $v$ accepts commit, $\varphi$ becomes CONFIRM.
    }  
\end{proof}


\begin{thm}
    If $\varphi = \text{CONFIRM}$, then $h$ is the highest ballot for which $v$ accepted \textit{commit}.
\end{thm}

\begin{proof}
    It suffices to show that, at any given moment,
    \begin{enumerate}[label=(\alph*)]
        \item % (a)
            $\varphi \ne \text{CONFIRM}$, or
        \item % (b)
            $h$ is the highest ballot for which $v$ accepted \textit{commit}.
    \end{enumerate}
    The invariant holds initially since (a) holds in the initial state.
    Suppose that the invariant holds in some state.
    There are $3 = 2^2 - 1$ ways in which this can be violated:
    \begin{enumerate}[label=(\roman*)]
        \item  % (i)
            Only (a) currently holds, and (a) becomes false.
        \item  % (ii)
            Only (b) currently holds, and (b) becomes false.
        \item  % (iii)
            Both (a) and (b) currently hold, and both become false.
    \end{enumerate}
    We will consider each possibility separately:
    \begin{enumerate}[label=(\roman*)]
        \item  % (i)
            Suppose that only (a) holds currently.
            In other words, $\varphi \ne \text{CONFIRM}$ and $h$ is not the highest ballot for which $v$ accepted \textit{commit}.
            \todo[inline,caption={}]{
                Finish this.
            }
        \item  % (ii)
            \todo[inline,caption={}]{
                Finish this.
            }
        \item  % (iii)
            \todo[inline,caption={}]{
                Finish this.
            }
    \end{enumerate}
\end{proof}

\begin{thm}
    If $h \ne \textbf{0}$, then $z = h.x$.
\end{thm}

\begin{proof}
    It suffices to show that, at any given moment,
    \begin{enumerate}[label=(\alph*)]
        \item % (a)
            $h = \textbf{0}$, or
        \item % (b)
            $z = h.x$.
    \end{enumerate}
    The invariant holds initially since (a) holds in the initial state.
    Suppose that the invariant holds in some state.
    There are $3 = 2^2 - 1$ ways in which this can be violated:
    \begin{enumerate}[label=(\roman*)]
        \item  % (i)
            Only (a) currently holds, and (a) becomes false.
        \item  % (ii)
            Only (b) currently holds, and (b) becomes false.
        \item  % (iii)
            Both (a) and (b) currently hold, and both become false.
    \end{enumerate}
    Since no action sets $h \leftarrow \textbf{0}$, (i) and (iii) are impossible.
    We will prove that (ii) is impossible.
    We modify $z$ and/or $h$ are Transition Type 1, and Step (2), (4), (6), and (7) in Transition Type 2.
    \begin{itemize}
        \item
            Transition Type 1: Since we are dealing with Case (ii), we assume $h \ne \textbf{0}$.
            Then $v$ will not perform Transition Type 1.
            In other words, $v$ will not alter the value of $z$ or $h.x$, so (b) cannot become false.
        \item
            Step (2) sets $z \leftarrow h.x$, so (b) holds after Step (2).
        \item
            Step (4) sets $z \leftarrow h.x$, so (b) holds after Step (4).
        \item
            Step (6)
            \todo[inline,caption={}]{
                It looks like I need to know that $b \sim h$.
                If I know that, I can say that $h' \sim b \sim h$, so $h' \sim h$.
            }
        \item
            Step (7)
            \todo[inline,caption={}]{
                It looks like the value doesn't change once we get to COMMIT?
                I'm not quite sure how I can prove this step.
            }
    \end{itemize}
\end{proof}

\begin{thm}
    $p = p' = \textbf{0}$ or $p' \lnsim p$.
\end{thm}

\begin{proof}
    $p = p' = \textbf{0}$ in the initial state, so this invariant holds in the initial state.
    We update $p, p'$ only in Step (1) and (5).
    \begin{itemize}
        \item
            After Step (1), the invariant must hold because we update $p, p'$ such that they are the two highest ballots accepted as prepared such that $p' \lnsim p$, where $p' = \textbf{0}$ or $p = p' = \textbf{0}$ if there are no such ballots.
            \begin{itemize}
                \item
                    If $p = p' = \textbf{0}$, then we are done.
                \item
                    If $p' = \textbf{0} \ne p$, then $p' \lnsim p$.
                \item
                    Otherwise, $p' \lnsim p$.
            \end{itemize}
        \item
            Step (5):
            \todo[inline,caption={}]{
                Interesting!
                I might need to prove other things first.
                $p \sim c$ makes me think that $c \ne \textbf{0}$, but it's not obvious to me!
            }
    \end{itemize}
\end{proof}

\begin{thm}
    If $p' \ne \textbf{0}$, then $p, p'$ are the two highest ballots accepted as prepared such that $p' \lnsim p$.
\end{thm}

\begin{proof}
    \todo[inline,caption={}]{
        prove this!
    }
\end{proof}
